\section*{P R E Á M B U L O} %Coloca un símbolo de de porcentaje antes de la \
%\lipsum[1] % Aquí iría el contenido del preámbulo
En la localidad de \localidad,
municipio de \municipio,
\estado;
siendo las \hora\ horas
del día \dia\ 
(\diadia),
del mes de \mes,
del año \anno\ 
(\annoanno);
dentro de las instalaciones
de la escuela primaria \escuela,
con CCT \cct,
turno \turno;
nos reunimos los C. C.,
\cargoa\ \profa\
(credencial para votar número \credenciala)
y \cargob\ \profb\
(credencial para votar número \credencialb)
quienes actúan en calidad de
Docente frente a grupo
y de Profesor de Educación Física,
respectivamente;
y, la directora del plantel,
\cargo\ \director\
(credencial para votar número \credencial),
en calidad de Autoridad Escolar,
para realizar el levantamiento de la presente acta
con el propósito de hacer constar los hechos
sobre las faltas de respeto
de la cual fue objeto la
\cargoa\ \profa\,
con respecto a
ser injuriada dentro del plantel
por el ya citado
\cargob\ \profb.\ \\

En la presente acta se exponen los hechos sucedidos
y las determinaciones a las que se llegaron,
manifestando en todo momento la voluntad de las partes
para poder realizarse esta acta
y, sobre todo,
para concluir
como se enuncia
con los malentendidos o faltas
que se pudieran derivar de la situación o situaciones
planteadas en las secciones siguientes.
Asimismo,
quienes intervienen en esta acta
manifiestan bajo protesta de decir verdad
que los hechos y declaraciones
contenidos en la misma son verídicos,
y que no existe falsedad alguna en su dicho.