% PROFR. RIGOBERTO VICENTE CASTRO.

% A CONTINUACIÓN DESCRIBO DETALLADAMENTE
% QUÉ HACE CADA UNA DE LAS LÍNEAS DEL CÓDIGO
% PORQUE COMPRENDER SU FUNCIONAMIENTO
% PERMITE SABER POR QUÉ SON APROPIADAS
% PARA ESTA ACTA CIRCUNSTANCIADA DE HECHOS.

% Tipo de documento: ARTÍCULO,
% es apropiado para este documento legal;
% papel tamaño carta
% y tipografía de 12 puntos para todo el documento.
\documentclass[letterpaper,12pt]{article}

% Margen de 2cm en cada lado de la página.
\usepackage[margin=2cm]{geometry}

% El siguiente paquete facilita la escritura en español,
% afecta a la separación de palabras,
% el formato de los encabezados
% y otros elementos específicos de este idioma.
\usepackage[spanish]{babel}

% Este paquete facilita
% la inclusión de CARACTERES ESPECIALES y ACENTOS
% directamente en este código
% para concentrarse en la redación.
\usepackage[utf8]{inputenc}

% Este paquete es necesario para utilizar la codificación T1,
% así se podrán usar caracteres acentuados y otros símbolos especiales.
% El paquete incluye caracteres como é, ñ, ü, etc.;
% también, su uso permite que la separación silábica se realice correctamente.
\usepackage[T1]{fontenc}

% Este paquete es para usar la fuente PALATINO.
% Palatino es una fuente serif clásica y legible;
% perfecta para documentos formales.
\usepackage{palatino}

% Se utiliza el siguiente paquete
% para personalizar la apariencia de las secciones,
% (en "Configuraciones iniciales" están los detalles).
\usepackage{titlesec}

% Con este paquete es posible crear las listas descriptivas
% para relatar ordenadamente
% los ANTECEDENTES, HECHOS, DECLARACIONES y DETERMINACIONES.
\usepackage{enumitem}

% Este paquete es necesario para agregar texto de relleno.
% Así se aprecia mejor como será el resultado final del documento.
\usepackage{lipsum}

% En la sección de FIRMAS
% este paquete demuestra su utilidad
% porque permite crear dos columnas
% para los firmantes del documento.
\usepackage{multicol} % Para el entorno multicolumnas

% En el documento se necesitan incluir
% secuencias numéricas largas
% como el número de la Credencial para Votar;
% este paquete permite la separación correcta
% de esos datos cuando están al final de la línea,
% así no se salen del margen.
\usepackage{seqsplit}

% Configuraciones iniciales
\titleformat{\section}{\Large\bfseries\centering}{\thesection}{1em}{}

% Datos del acta
\newcommand{\localidad}{LOCALIDAD}
\newcommand{\municipio}{MUNICIPIO}
\newcommand{\estado}{ESTADO}
\newcommand{\hora}{00:00}
\newcommand{\dia}{29}
\newcommand{\diadia}{VEINTINUEVE}
\newcommand{\mes}{MES}
\newcommand{\anno}{2024}
\newcommand{\annoanno}{DOS MIL VEINTICUATRO}
\newcommand{\escuela}{"HÉROE DE LA PATRIA"}
\newcommand{\cct}{30EPR1234Q}
\newcommand{\turno}{TURNO}
\newcommand{\director}{NOMBRE1 PATERNO1 MATERNO1}
\newcommand{\cargo}{profesora}
\newcommand{\credencial}{\seqsplit{1234567890123}}
\newcommand{\profa}{NOMBRE1 PATERNO1 MATERNO1}
\newcommand{\cargoa}{profesora}
\newcommand{\credenciala}{\seqsplit{1234567890123}}
\newcommand{\profb}{NOMBRE2 PATERNO2 MATERNO2}
\newcommand{\cargob}{profesor}
\newcommand{\credencialb}{\seqsplit{1234567890123}}
%\newcommand{\profc}{NOMBRE3 PATERNO3 MATERNO3}
%\newcommand{\cargoc}{profesor}
%\newcommand{\credencialc}{\seqsplit{1234567890123}}
%\newcommand{\profd}{NOMBRE4 PATERNO4 MATERNO4}
%\newcommand{\cargod}{profesor}
%\newcommand{\credenciald}{\seqsplit{1234567890123}}

% Aquí comienza el ACTA DE HECHOS.
\begin{document}

% ENCABEZADO del documento
% con fecha del acta visible.
\begin{center}
    \textbf{\Large ACTA CIRCUNSTANCIADA DE HECHOS}\\
    \vspace{12pt}
    \textbf{\dia\ DE \mes\ DE \anno}\\
    \vspace{12pt}
\end{center}

% Incluir archivos de secciones
\section*{P R E Á M B U L O} %Coloca un símbolo de de porcentaje antes de la \
%\lipsum[1] % Aquí iría el contenido del preámbulo
En la localidad de \localidad,
municipio de \municipio,
\estado;
siendo las \hora\ horas
del día \dia\ 
(\diadia),
del mes de \mes,
del año \anno\ 
(\annoanno);
dentro de las instalaciones
de la escuela primaria \escuela,
con CCT \cct,
turno \turno;
nos reunimos los C. C.,
\cargoa\ \profa\
(credencial para votar número \credenciala)
y \cargob\ \profb\
(credencial para votar número \credencialb)
quienes actúan en calidad de
Docente frente a grupo
y de Profesor de Educación Física,
respectivamente;
y, la directora del plantel,
\cargo\ \director\
(credencial para votar número \credencial),
en calidad de Autoridad Escolar,
para realizar el levantamiento de la presente acta
con el propósito de hacer constar los hechos
sobre las faltas de respeto
de la cual fue objeto la
\cargoa\ \profa\,
con respecto a
ser injuriada dentro del plantel
por el ya citado
\cargob\ \profb.\ \\

En la presente acta se exponen los hechos sucedidos
y las determinaciones a las que se llegaron,
manifestando en todo momento la voluntad de las partes
para poder realizarse esta acta
y, sobre todo,
para concluir
como se enuncia
con los malentendidos o faltas
que se pudieran derivar de la situación o situaciones
planteadas en las secciones siguientes.
Asimismo,
quienes intervienen en esta acta
manifiestan bajo protesta de decir verdad
que los hechos y declaraciones
contenidos en la misma son verídicos,
y que no existe falsedad alguna en su dicho.
\section*{A N T E C E D E N T E S}
\begin{description}
    \item[ÚNICO.-] \lipsum[1]
    %\item[PRIMERO.-] \lipsum[1]
    %\item[SEGUNDO.-] \lipsum[2]
    %\item[TERCERO.-] \lipsum[3]
    %\item[CUARTO.-] \lipsum[4]
    %\item[QUINTO.-] \lipsum[5]
    % \item[SEXTO.-] \lipsum[6]
    % \item[SÉPTIMO.-] \lipsum[7]
    % \item[OCTAVO.-] \lipsum[8]
    % \item[NOVENO.-] \lipsum[9]
    % \item[DÉCIMO.-] \lipsum[10]
    % \item[UNDÉCIMO.-] \lipsum[11]
    % \item[DUODÉCIMO.] \lipsum[12]
    % \item[DECIMOTERCERO.-] \lipsum[13]
    % \item[DECIMOCUARTO.-] \lipsum[14]
    % \item[DECIMOQUINTO.-] \lipsum[15]
    % \item[DECIMOSEXTO.-] \lipsum[16]
    % \item[DECIMOSÉPTIMO.-] \lipsum[17]
    % \item[DECIMOCTAVO.-] \lipsum[18]
    % \item[DECIMONOVENO.-] \lipsum[19]
    % \item[VIGÉSIMO.-] \lipsum[20]
\end{description}
\section*{H E C H O S}
\begin{description}
    \item[PRIMERO.-] \lipsum[1]
    \item[SEGUNDO.-] \lipsum[2]
    \item[TERCERO.-] \lipsum[3]
    \item[CUARTO.-] \lipsum[4]
    \item[QUINTO.-] \lipsum[5]
    % \item[SEXTO.-] \lipsum[6]
    % \item[SÉPTIMO.-] \lipsum[7]
    % \item[OCTAVO.-] \lipsum[8]
    % \item[NOVENO.-] \lipsum[9]
    % \item[DÉCIMO.-] \lipsum[10]
    % \item[UNDÉCIMO.-] \lipsum[11]
    % \item[DUODÉCIMO.] \lipsum[12]
    % \item[DECIMOTERCERO.-] \lipsum[13]
    % \item[DECIMOCUARTO.-] \lipsum[14]
    % \item[DECIMOQUINTO.-] \lipsum[15]
    % \item[DECIMOSEXTO.-] \lipsum[16]
    % \item[DECIMOSÉPTIMO.-] \lipsum[17]
    % \item[DECIMOCTAVO.-] \lipsum[18]
    % \item[DECIMONOVENO.-] \lipsum[19]
    % \item[VIGÉSIMO.-] \lipsum[20]
\end{description}
\section*{D E C L A R A C I O N E S}
\begin{description}
    \item[PRIMERA.-] \lipsum[1]
    \item[SEGUNDA.-] \lipsum[2]
    \item[TERCERA.-] \lipsum[3]
    %\item[CUARTA.-] \lipsum[4]
    %\item[QUINTA.-] \lipsum[5]
    %\item[SEXTA.-] \lipsum[6]
    %\item[SÉPTIMA.-] \lipsum[7]
    %\item[OCTAVA.-] \lipsum[8]
    %\item[NOVENA.-] \lipsum[9]
    %\item[DÉCIMA.-] \lipsum[10]
    %\item[UNDÉCIMA.-] \lipsum[11]
    %\item[DUODÉCIMA.-] \lipsum[12]
    %\item[DECIMOTERCERA.-] \lipsum[13]
    %\item[DECIMOCUARTA.-] \lipsum[14]
    %\item[DECIMOQUINTA.-] \lipsum[15]
    %\item[DECIMOSEXTA.-] \lipsum[16]
    %\item[DECIMOSÉPTIMA.-] \lipsum[17]
    %\item[DECIMOCTAVA.-] \lipsum[18]
    %\item[DECIMONOVENA.-] \lipsum[19]
    %\item[VIGÉSIMA.-] \lipsum[20]
\end{description}
\section*{D E T E R M I N A C I O N E S}
\begin{description}
    \item[PRIMERA.-] \lipsum[1]
    \item[SEGUNDA.-] \lipsum[2]
    \item[TERCERA.-] \lipsum[3]
    %\item[CUARTA.-] \lipsum[4]
    %\item[QUINTA.-] \lipsum[5]
    %\item[SEXTA.-] \lipsum[6]
    %\item[SÉPTIMA.-] \lipsum[7]
    %\item[OCTAVA.-] \lipsum[8]
    %\item[NOVENA.-] \lipsum[9]
    %\item[DÉCIMA.-] \lipsum[10]
    %\item[UNDÉCIMA.-] \lipsum[11]
    %\item[DUODÉCIMA.-] \lipsum[12]
    %\item[DECIMOTERCERA.-] \lipsum[13]
    %\item[DECIMOCUARTA.-] \lipsum[14]
    %\item[DECIMOQUINTA.-] \lipsum[15]
    %\item[DECIMOSEXTA.-] \lipsum[16]
    %\item[DECIMOSÉPTIMA.-] \lipsum[17]
    %\item[DECIMOCTAVA.-] \lipsum[18]
    %\item[DECIMONOVENA.-] \lipsum[19]
    %\item[VIGÉSIMA.-] \lipsum[20]
\end{description}
\section*{F I R M A S}
\begin{multicols}{2}
    \begin{center}
        \rule{7cm}{0.4pt} \\[0.2cm]
        Profr. Nombre Paterno Materno \\ 
        Director del Plantel Educativo
    \end{center}

    \begin{center}
        \rule{7cm}{0.4pt} \\[0.2cm]
        Profra. Nombre Paterno Materno \\ 
        Docente de grupo
    \end{center}

    \begin{center}
        \rule{7cm}{0.4pt} \\[0.2cm]
        Profr. Nombre Paterno Materno \\ 
        Docente de grupo
    \end{center}

    \begin{center}
        \rule{7cm}{0.4pt} \\[0.2cm]
        Profra. Nombre Paterno Materno \\ 
        Docente de grupo
    \end{center}

    \begin{center}
        \rule{7cm}{0.4pt} \\[0.2cm]
        Profr. Nombre Paterno Materno \\ 
        Docente de grupo
    \end{center}

    \begin{center}
        \rule{7cm}{0.4pt} \\[0.2cm]
        Profra. Nombre Paterno Materno \\ 
        Docente de grupo
    \end{center}

    % Mueve el comando \columnbreak entre las firmas cuando el número de firmantes es impar
    % Por ejemplo, si hay 5 firmantes, separa el contenido en tres firmas arriba
    % y dos firmas abajo de aquí
    % \columnbreak % ELIMINA EL SÍMBOLO DE PORCENTAJE CUANDO USES ESTE COMANDO Y EL ESPACIO
    
\end{multicols}

\end{document}